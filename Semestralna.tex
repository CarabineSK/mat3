% !TeX spellcheck = sk_SK
\documentclass[a4paper,10pt]{article}
\usepackage[slovak]{babel}
\usepackage[utf8]{inputenc}
\usepackage{amsmath}
\usepackage{amsthm}
\usepackage{setspace}
\usepackage{graphicx}

\newcommand{\HRule}{\rule{\linewidth}{0.5mm}}

\theoremstyle{plain}
\newtheorem{thm}{Theorem}[section]

\theoremstyle{definition}
\newtheorem{defin}[thm]{Definícia}

\begin{document}

\begin{titlepage}
\begin{center}

\includegraphics[scale=2,keepaspectratio=true]{./FRI_RGB_SK.png}
 % FRI_RGB_SK.png: 200x200 pixel, 300dpi, 1.69x1.69 cm, bb=0 0 48 48

\textsc{\newline\LARGE Žilinská univerzita v Žiline}\\[0.5cm]

\textsc{\Large Fakulta riadenia a informatiky}\\[0.5cm]

% Title
\HRule \\[0.4cm]
{ \huge \bfseries Analýza diferenčnej rovnice \\[0.4cm] }

\HRule \\[1.5cm]

% Author and supervisor
\begin{minipage}{0.4\textwidth}
\begin{flushleft} \large
\emph{Autori:}\\
Matejko Peter\\
Mudrák Ľuboš\\
Rehák Tomáš\\
Zárecký Martin\\
Boďa Michal\\
Kapusta Peter
\end{flushleft}
\end{minipage}
\begin{minipage}{0.4\textwidth}
\begin{flushright} 
\end{flushright}
\end{minipage}

\vfill

% Bottom of the page
{\large \today}

\end{center}
\end{titlepage}


\newpage

\tableofcontents



\newpage
\section{Úvod}

Táto práca analyzuje riešenia diferenčnej rovnice 
\begin{equation}
x_{n+1}=\left(a+{\frac{b}{n}}\right)\,x_{n}
\end{equation}
v závislosti od hodnôt $ a $ a $ b $, kde $ a \text{ a } b $ sú reálne čísla, také, že $ a + b > 0$.
%doplnit
\section{Definície}
\subsection{Pojem diferencia}

\begin{defin}
Je daný bod $x_{0}$ a číslo $h > 0$. Nech funkcia $y = f(x)$ je 
definovaná v bodoch $x_{0} \text{ a } x_{0} + h$. \textit{Diferencia funkcie }$f(x)$\textit{ v bode }$x_{0}$
je číslo $f(x_{0} + h) - f(x_{0})$. Značíme
$$\Delta f(x_{0}) = f(x_{0} + h) - f(x_{0})$$
\end{defin}
\subsection{Pojem diferenčná rovnica}
\begin{defin}
(Diferenčné rovnice 1. typu) 
Nech pre všetky $x \in M$ je
definovaná funkcia $f(x, y, \Delta y, \Delta ^{2}y, \ldots, \Delta ^{k}y)$. Rovnica tvaru
$$f(x, y, \Delta y, \Delta ^{2}y, \ldots, \Delta ^{k}y) = 0 \text{ ,}$$
v ktorej neznámou funkciou $y = \varphi(x)$, nazývame diferenčnú rovnicu k-tého 
rádu a 1. typu definovanú v $M$.
\end{defin}

\textbf{\textit{Partikulárnym riešením}} tejto rovnice v $M$ nazveme každú funkciu
$y = \varphi(x)$, ktorá pre všetky $x \in M$ spĺňa danú rovnicu.

\paragraph{}
\textbf{\textit{Všeobecným riešením}} nazývame množinu všetkých partikulárnych
riešení.

\paragraph{}
\begin{defin}
(Diferenčné rovnice 2. typu)
Nech je pre všetky $x \in M$
definovaná funkcia
$$g(x, y_{x}, y_{x+1}, \ldots, y_{x+k})\text{, kde }y_{x+j} = \varphi(x + j) j = 0, 1, 2, \ldots, k\text{.}$$
\end{defin}

Rovnicu tvaru
$$g(x, y_{x}, y_{x+1}, \ldots, y_{x+k}) = 0\text{,}$$
v ktorej neznáma funkcia $y_{x} = \varphi(x)$, nazývaná \textit{diferenčná rovnica 2.typu}
definovaná v $M$. Ak je závislosť $g$ na $y_{x}$ a $y_{x+k}$ nekonštantná hovoríme,
že rovnica \textit{je k-tého rádu}. Riešenie rovnice v $M$ nazývame každú funkciu
$y_{x} = \varphi(x)$, ktorá pre všetky $x \in M$  spĺňa danú rovnicu. K tomu je nutné,
aby definičný obor funkcie $\varphi(x)$ obsahoval všetky $x \in M$ a taktiež body
$x + 1, x + 2, \ldots, x + k$.
\newpage
\subsubsection{Rekurentná formula}
Rekurentnú formulu vieme získať z diferenčnej rovnice vyjadrením $(n+k)$-tého člena pomocou $k$ predchádzajúcich členov rovnice.


\indent Majme danú diferenčnú rovnicu: $$g(x, y_{x}, y_{x+1}, \ldots, y_{x+k}) = 0\text{.}$$

\begin{enumerate}
	\item Nech definičný obor tejto rovnice sú prirodzené čísla $ n = 1, 2, 3, \ldots$ a
		ďalej zaveďme všeobecnejšie označenie pre členy postupnosti: $ y_{n} =  a_{n}$,
		takže rovnicu vieme prepísať do tvaru  $$g(n, a_{n}, a_{n+1}, \ldots, a_{n+k}) = 0\text{.}$$
		Predpokladajme, že túto rovnicu vieme jednoznačne rozriešiť vzhľadom k $ a_{n+k}$ :
		$$a_{n+k} = G(n, a_{n}, a_{n+1}, \ldots, a_{n+k-1})\text{,}$$
		kde $G$ je funkcia, ktorú sme dostali riešením pôvodnej rovnice.\linebreak[4]
		Týmto sme dostali všeobecný rekurentný vzorec pre postupnosť $a_{n}$,\linebreak[4] v ktorom je $(n+k)$-ty 				člen vyjadrený pomocou $k$ predchádzajúcich členov 
		$a_{n}, a_{n+1}, \ldots, a_{n+k-1}$ a premennej $n$.
	\item Pozrime sa na riešenie, keď máme vopred dané (ľubovoľné) čísla $ a_{1}, a_{2}, \ldots, a_{k}$.
		Vieme, že po dosadení členov do funkcie $G$ vypočítame jednoznačne člen
		$$a_{k+1} = G(1, a_{1}, a_{2}, \ldots, a_{k})\text{,}$$ ďalším dosadením vypočítame
		$$a_{k+2} = G(2, a_{2}, a_{3}, \ldots, a_{k+1})\text{atď.}$$
		Všeobecný $n$-tý člen $ a_{n} $ dostaneme vypočítaním elementárnej funkcie
		$n$ a daných $k$ prvých čísiel $ a_{1}, a_{2}, \ldots, a_{k}$. Táto funkcia je práve partikulárnym riešením 				diferenčnej rovnice s počiatočnými podmienkami $ a_{1}, a_{2}, \ldots, a_{k}$.	
\end{enumerate}

Touto druhou úvahou sa súčasne znovu potvrdzuje, že všeobecné riešenie rovnice $k$-tého rádu 
$a_{n+k} = G(n, a_{n}, a_{n+1}, \ldots, a_{n+k-1})$ má obsahovať
$k$ všeobecných konštánt, ktoré je možno si ľubovoľne zvoliť.

\newpage
\section{Vypracovanie}
\textbf{Popis problému}\newline
Analýzou diferenčnej rovnice $ (1) $ sme zistili, že sa jedná o homogénnu lineárnu diferenčnú rovnicu prvého stupňa druhého typu s nekonštantnými koeficientami. Ďalej sa budeme zaoberať riešeniami tejto rovnice s využitím poznatku, že sa jedná o rekurentný popis postupnosti. Dosadzovaním rôznych hodnôt za $ a $ a $ b $ budeme hľadať riešenia rovnice, a teda priameho vzorca pre členy postupnosti. \newline\newline
Prvých pár členov postupnosti má tvar : 
\begin{spacing}{1.4}
\begin{tabbing}
\hspace{2cm}\=\kill
 $n=1$ :\> $x_{2}= (a+b)\*x_{1}$\\ 
 $n=2$ :\>  $x_{3}=(a+\frac{b}{2})\*x_{2}=(a+\frac{b}{2})(a+b)\*x_{1}$ \\ 
 $n=3$ :\>$x_{2}=(a+\frac{b}{3})\*x_{3}=(a+\frac{b}{3})\*(a+\frac{b}{2})(a+b)\*x_{1}$\\
 $n=4$ :\> $x_{2}=(a+\frac{b}{4})\*x_{4}=(a+\frac{b}{4})\*(a+\frac{b}{3})\*(a+\frac{b}{2})
 (a+b)\*x_{1}$\\ 
  $n=k-1$ :\> $x_{k}=(a+\frac{b}{k-1})\*x_{k-1}=(a+\frac{b}{k-1})\*(a+\frac{b}{k-2})\ldots(a+\frac{b}{2})\*(a+b)\*x_{1}$
\end{tabbing} 
\end{spacing}
 

\subsection{Triviálne riešenie}
Nech $x_{1}=0$.\newline
 Potom dostávame triviálne riešenie $x_{n+1} = 0$ také, že každý člen postupnosti bude mať hodnotu $0$. Ďalej budeme predpokladať, že $ x_{1} > 0 $, a teda sa budeme 
zaoberať závislosťou od hodnôt $ a,b $.

\subsection{Všeobecné riešenie}
$x_{n+1}=\left(a+{\frac{b}{n}}\right)\*x_{n} \text{, } n \geq 1 \text{ a nech } a + {\frac{b}{n}} = f(n)$, potom
$x_{n+1}=f(n)\*x_{n}   $, kde $ n = 1 + k $, $ k = 0,1,2,... $
\begin{align*}
x_{n}&=f(n-1)\*x_{n-1}    \\
x_{n-1}&=f(n-2)\*x_{n-2}    \\
x_{n-2}&=f(n-3)\*x_{n-3}    \\
&\vdots \\
x_{n-(k-1)}&=f(n-k)\*x_{n-k}\text{,}    
\end{align*}


\noindent a teda $ x_{2} = f(1)\*x_{1} $. 
%Kde počiatočná hodnota $x_{n}$ v bode $n=1$ je prvý člen postupnosti, teda zvolená konštanta.%
\newpage
\subsection{Závislosť od hodnôt a, b}

Predpokladajme, že $ b=0 $ a $ a > 0 $, teda rovnica $ (1) $ má tvar $ x_{n+1} = a\,x_{n} $, čiže sa jedná o \textbf{geometrickú postupnosť}
$ x_{n} = a\*x_{n-1} $, kde $ a $ je koeficientom geometrickej postupnosti. V prípade, že $ a = 1 $ dostávame $ x_{n} = x_{n-1} $.
%Vraj nejake grafy%
\\Ďalej skúmajme prípad, kedy je $ a=0 $. Prvých pár členov vyzerá následovne: 
\begin{align*}
x_{2}&=b\*x_{1}\\
x_{3}&=\frac{b}{2}\*x_{2} = \frac{b}{2}\*b\*x_{1}=\frac{b^{2}}{2}\*x_{1} \\
x_{4}&=\frac{b}{3}\*x_{3} = \frac{b^{3}}{3!}\*x_{1}  \\
&\vdots \\
x_{n}&=\frac{b^{n-1}}{(n-1)!}\*x_{1}    \\
x_{n+1}&=\frac{b^{n}}{(n)!}\*x_{1}   \\
\end{align*}
Čo sa podobá na Poissonovo rozdelenie pravdepodobnosti. \\
Položme $ \sum\limits_{n=0}^\infty x_{n+1} = 1 $, potom $ 1=\sum\limits_{n=0}^\infty x_{n+1} = x_{1}\*\sum\limits_{n=0}^\infty \frac{b^{n}}{n!} = x_{1}\*e^{b}$ $ \Rightarrow $ $ x_{1} = e^{-b} $.
Môžeme teda prehlásiť, že ak $x_{1}=e^{-b} $, potom riešením rovnice\\
$(1)$ kde $a=0$, $b>0$
je \textbf{Poissonovo rozdelenie} $Po(\lambda)$ s parametrom $\lambda=b$.\\
Metódou matematickej indukcie sme teda dokázali, že $x_{n}\sim Po(b)$, keďže každý ďalší člen $x_{n+1}\sim Po(b)$, resp. $ X \sim Po(b)$.\\
%Koniec 246%%247%
\\Ďalej skúmajme prípad, kedy $ a \in (0;1) $. Z pôvodnej rovnice teda :
\begin{align*}
x_{2}&=(a+b)\*x_{1} \\
x_{3}&=(a + \frac{b}{2}) \* (a + b)\*x_{1} \\
x_{4}&=(a + \frac{b}{3})\*x_{3}= (a + \frac{b}{3})\*(a + \frac{b}{2})\*(a + b)\*x_{1} \\
&\vdots \\
x_{n}&=(a + \frac{b}{n-1})\*(a + \frac{b}{n-2}) \ldots (a + \frac{b}{2})\*(a + b)\*x_{1}\\
x_{n+1}&=(a + \frac{b}{n})\*(a + \frac{b}{n-1}) \ldots (a + \frac{b}{2})\*(a + b)\*x_{1}\\
&=\frac{a^{n}}{n!}\*(n + \frac{b}{a})\*(n-1 + \frac{b}{a}) \ldots (2 + \frac{b}{a})\*(1 + \frac{b}{a})\*x_{1},
\end{align*}
\noindent kde $ x_{1} $ je prvým členom postupnosti. \newpage
\noindent Vychádzajúc z $ a + b > 0 $ nech $ 1 + \frac{b}{a} = \alpha > 0$\\
$$ x_{n+1} = \frac{a^{n}}{n!}\*(n+\alpha-1)\*(n+\alpha-2) \ldots (1+\alpha)\*\alpha\*x_{1}$$\\
Po rozšírení pravej strany jednotkou v tvare $ \frac{(\alpha-1)!}{(\alpha-1)!} $ dostávame : 
\begin{align*}
x_{n+1} &= \frac{a^{n}\*(n+\alpha-1)!}{n!\*(\alpha-1)!}\*x_{1} \\
x_{n+1} &= a^{n}\* \dbinom{n+\alpha-1}{n}\*x_{1} \\	%Pr alebo Po Kenny dobre ?%
x_{n+1} &= \dbinom{n+\alpha-1}{n}\*(1-a)!^{\alpha}\*a^{n} = Pr(X = n) ,
\end{align*}
\noindent kde $ x_{1} = (1-a)^{\alpha} \Rightarrow x_{1} = (1-a)^{1+\frac{b}{a}}$.
%Koniec 247%%Teraz 248%
Môžeme teda povedať, že $ x_{n+1} \sim NB(2,9) $, čiže $ x_{n+1}\sim NB (1+\frac{b}{a}; a) $, čo je \textbf{negatívne binomické rozdelenie}, pričom
$ a \in (0;1), a+b>0 \Rightarrow b \in (0;\infty), x_{1}=(1-a)^{a+\frac{b}{a}}$. Matematickou dedukciou je teda dokázané, že to platí aj pre 
$ x_{n} $.\\
%Koniec 248%%Zacina 244, ale to len prepisem, lebo neviem kam s tym%
%Keny mi prave povedal ze to tam nemam davat, co som napisal som zakomentoval 
%Ďalej budeme skúmať ďalší prípad, kedy $ a>0 $.\\
%Citujem "[21:28:33] Peter Kapusta: ten 244 tam nedavaj
%[21:28:42] Peter Kapusta: to zaro picovinu odfotil
%Z pôvodnej rovnice teda : 
%\begin{spacing}{1.6}
%\begin{tabbing}
%\hspace{0.7cm}\=\kill 
%$ x_{2}$ \> $=(a+b)\*x_{1} $\\
%$ x_{3}$ \> $=(a + \frac{b}{2}) \* (a + b)\*x_{1} ) $\\
%$ x_{4}$ \> $=(a + \frac{b}{3})\*x_{3}= (a + \frac{b}{3})\*(a + \frac{b}{2})\*(a + b)\*x_{1} $\\
%\end{tabbing} 
%\end{spacing} %
%Peto screen 46

Následne je potrebné sa zaoberať prípadom, keď $ a<0 $. Nesmieme však zabudnúť na podmienku $ a+b=0 $.
Predpokladajme, že existuje kladné celé číslo $ z $ také, že $ a + \frac{b}{z+1} = 0 $. V tomto prípade
platí, že pre všetky $ n\geq z+1  $ $ x_{n+1} = 0 $.
Keďže $ \frac{b}{n} \rightarrow 0$ a $ b>0 $, $a + \frac{b}{n} \leq 0$ pre všetky dostatočne veľké $ n $. Ak také $ z $ neexistuje, potom zvolením minimálneho $ n $, takého, že $ a + \frac{b}{n} < 0 $ dostávame $ x_{n+1}<0 $, čo je spor.
Môžeme teda písať $ z = -(1+\frac{b}{a}) $, potom :
\begin{align*}
x_{2}&=(a+b)\*x_{1} \\
x_{3}&=(a + \frac{b}{2}) \* (a + b)\*x_{1}\\
&\vdots \\
x_{n} &= (a+\frac{b}{n-1})\*(a+\frac{b}{n-2})\ldots  (a + \frac{b}{2})\*(a+b)\*x_{1}\\
x_{n+1} &= (a + \frac{b}{n})\*(a + \frac{b}{n-1})\ldots (a + \frac{b}{2})\*(a+b)\*x_{1}\\
&= \frac{a^{n}}{n!}\*(n + \frac{b}{a})\*(n-1+\frac{b}{a}) \ldots (2+\frac{b}{a})\*(1+\frac{b}{a})\*x_{1} \\
&= \frac{a^{n}}{n!}\*(-z+n-1)\*(-z+n-2) \ldots (-z+1)\*(-z)\*x_{1} \\
&= (-1)^{n}\*\frac{a^{n}}{n!}\*(n-1-z)\*(n-2-z)\ldots(1-z)\*z*x_{1} \\
&= \dbinom{z}{n}\*(-a)^{n}\*x_{1}  \\
\end{align*}\newpage
\noindent Nech je $ A = -a >0 $. Ak $ \sum\limits_{n=0}^\infty x_{n+1} = 1 $ a $ x_{n+1} = 0 $ pre $ n\geq z+1 $ potom 
$$ x_{1} \sum\limits_{n=0}^z \dbinom{z}{n} A^{n}= 1 \text{.}$$
\noindent Podľa binomickej vety, $ \sum\limits_{n=0}^z \dbinom{z}{n} A^{n}= (1+A)^{z} $, potom $ x_{1} = (1+A)^{-z} $.
Keďže každé kladné číslo môžeme zapísať ako $ A = \frac{p}{1-p} $, kde $ p \in (0,1) $ dostávame: 
$$ x_{1} = (1 + \frac{p}{1-p})^{-z} = (1-p^{z}) \Rightarrow x_{n+1} = \dbinom{z}{n}\*p^{n}\*(1-p^{z-n}) $$ a to 
môžeme zapísať ako: $$ X \sim Bin(z,\frac{a}{a-1}) \text{.}$$%Peto screen 46 koniec
\newpage %Kecy z doc
\section{Využitie rozdelení pravdepodobností\\
 v poisťovníctve a teórii rizika} %Doc file stuff !!! -.-
Výskyt poistných plnení je náhodný proces a teda nie je možné poznať ako sa bude vyvíjať. Pre poisťovňu je dôležité tento proces aspoň predpovedať, pretože na základe toho musí stanoviť výšku poistného pre jednotlivé zmluvy a takisto výšku rezerv. 
Súhrn strát ktoré poisťovňa utrpí môžeme vyjadriť vzorcom:
$$ S(t) = \sum\limits_{j=1}^N(t) X_{j},\; \forall t\geq0 $$\\
\textit{S(t)} – súhrn strát \\
\textit{X(j)} – náhodná premenná predstavujúca výskyt poistnej udalosti, náhodné premenné X(j) sú nezávislé, ale vieme o nich, že sú  z jedného rozdelenia pravdepodobnosti\\
\textit{N(t)} – diskrétna náhodná premenná\\

Pri predpovedaní výpočtu poistného je dôležité nájsť vhodné rozdelenie pre počet poistných plnení a takisto rozdelenie výšky poistných plnení. Jednotlivé poistné udalosti sú náhodné a je možné predpokladať aj ich nezávislosť pre konkrétne zmluvy. V prípade počtu plnení sú známe diskrétne rozdelenia, ktoré sú na ich popisovanie najvhodnejšie.
\subsection{Binomické rozdelenie $ Bi(n,p) $}
Opisuje počet výskytu určitej náhodnej udalosti v $ n $ nezávislých pokusoch, pričom daný jav má stále rovnakú pravdepodobnosť $ p $.

Náhodná premenná $ N $ má binomické rozdelenie pravdepodobnosti s parametrami $ n; p $ práve vtedy, ak pravdepodobnosť, že pri $ n $ nezávislých pokusoch nastane pozorovaný jav práve $ k $ krát, má tvar: 
$$ p_{N}(k)=\dbinom{n}{k}\*q^{k}\*(1-q)^{n-k} \text{, pre }k=1,2,\ldots,n\text{.} $$
\subsection{Poissonovo rozdelenie $ Po(\lambda) $}
Toto rozdelenie používame na aproximáciu málo pravdepodobných náhodných udalostí pri veľkom počte nezávislých opakovaní experimentu.

Náhodná premenná $ N $ má Poissonovo rozdelenie s parametrom $ \lambda $, práve vtedy, ak pravdepodobnostná funkcia má tvar:
$$ p_{N}(k)=\frac{\lambda^{k}}{k!}\*e^{-\lambda},\; \text{pre }k =0,1,2,\ldots .$$
\newpage\subsection{Geometrické rozdelenie $ Ge(p) $}
Pri tomto rozdelení nezávisle opakujeme pokus, pričom pravdepodobnosť, že náhodná udalosť nastane, je $ p $. Pokus budeme opakovať toľkokrát, kým prvýkrát nastane daná udalosť. Náhodná premenná $ N $ teda bude predstavovať počet opakovaní, kým nastala pozorovaná udalosť.
Náhodná premenná $ N $ má geometrické rozdelenie s parametrom $ p $ práve vtedy, ak pravdepodobnostná funkcia má tvar:
$$ p_{N}(k)=p\*(1-p)^{k}\text{,}\;\text{pre }k=0,1,2,\ldots .$$
\subsection{Negatívne binomické rozdelenie $ NBi(m,p) $}
Toto rozdelenie vychádza z geometrického rozdelenia, teda znova nezávisle opakujeme
pokusy, pričom tentokrát neskončíme vtedy, keď náhodná udalosť nastane prvýkrát, ale až vtedy, keď nastane $ m $-krát.
Náhodná premenná $ N $ má negatívne binomické rozdelenie pravdepodobnosti s parametrami $ m $ a $ p $ práve vtedy, 
ak jej pravdepodobnostná funkcia má tvar:
$$ p_{N}(k) = \dbinom{k+m-1}{m-1}\*p^{m}\*(1-p)^{k}\text{,}\; \text{pre }k=0,1,2,\ldots \text{.}$$
%Koniec prepisu z doc
\newpage
\section{Záver}
\hfill{Touto  prácou sme sa dopracovali k riešeniam rovnice, ktorá rekurentným\\ spôsobom popisuje postupnosť čísel. Zdanlivo nenápadná rovnica však našla svoje riešenia v teórii pravdepodobnosti, a to len vhodným dosadzovaním jednotlivých konštánt. Ukázalo sa, že jednotlivé rozdelenia pravdepodobnosti v nej obsiahnuté nachádzajú svoje uplatnenie v štatistike, a tým aj v teórii rizika a poisťovníctve.}

\begin{quote}
\textit{Jednou z pozoruhodných vecí na javoch nášho sveta je, s akou neobyčajnou presnosťou mu vládnu matematické zákony.}

\hfill R. Penrose 

\end{quote}

\begin{thebibliography}{9}
\bibitem{pra}{\em Prágerová, A.:}
               {\bf Diferenční rovnice.}
           Polytechnická knižnice, Praha 1971.
\end{thebibliography}

\end{document}
